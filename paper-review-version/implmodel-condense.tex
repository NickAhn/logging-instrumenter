\section{A Brief Review of the Instrumentation Model} \label{sec:implmodel}

In \cite{X}, we have explored the full formalization of the implementation model for instrumenting concurrent systems that guarantee the correctness of audit logs, according to logging specifications that go beyond Horn clauses. In this section, we review this model briefly. 

\subsection{Source System Model} \label{sec:pi}

The source system $\srclang$ is concurrent program, which is modeled in $\pi$-calculus \cite{parrow2001introduction}. Top-level components of $\srclang$, denoted by $A$, are called top-level agents. Top-level agents execute in parallel, and communicate among themselves as their functionality dictates. Some $A$ consists of internal modules and/or functions, modeled as subagents $B^A$. As part of the definition of the source system, we assume the existence of a codebases $\codebaseu$ and $\codebasel$ that include definitions of the form $A(x_1, \cdots, x_n) \triangleq \cdots$ and $B^A (x_1, \cdots, x_n) \triangleq \cdots$ resp.



\subsection{A Class of Logging Specifications} \label{sec:logspec}
Horn clause logic has been used to specify audit logging requirements in previous work \cite{X}, where certain events must take place so that audit logs are generated. In this paper, we go beyond Horn clauses in our implementation model to help specify not only the necessity of having certain events to occur, but also to ensure that another group of events do not take place. In this respect, we define a class of specifications that assert temporal relations among the events that must or must not transpire in different components (top-level agents) of the system. Using this extended class of specifications, a particular event must be logged, provided that a certain set of events have taken place and another set of events have not. We would call these different sets of events positive and negative triggers, resp. We use $\callpols$ to denote this class of specifications. In $\callpols$, each event is modeled as a sub-agent invocation, i.e., a module within one of the agents of the system. Figure \ref{fig:callpols} depicts the structure of specifications in $\callpols$. $\folp{Call}(t,A,B,mathit{xs})$ asserts the event of invoking sub-agent $B^A$ at time $t$ with list of parameters $\mathit{xs}$. $\phi$ and $\psi_j$ are possibly empty conjunctive sequence of literals of the form $t_i < t_j$. $A_0$ is called a \emph{logging event agent}, whereas other $A_i$s are called \emph{positive trigger agents}. $A'_j$s are called \emph{negative trigger agents}. Similarly, \emph{logging event sub-agent} refers to $B_0$, and other $B_i$s are called \emph{positive trigger sub-agents}. $B'_j$s are called \emph{negative trigger sub-agents}. \emph{Logging preconditions} are predicates $\folp{Call}(t_i,A_i,B_i,\tilde{x})$ for all $i \in \{1,\cdots, n\}$ and $\folp{Call}(t_j , A'_j , B'_j , \mathit{ys}_j )$ for all $j \in \{1, \cdots, m\}$. 

For example, in the microservices-based MRS, described in Figure \ref{X}, each microservice, including Authorization and Patient, is a top-level agent in $\srclang$. Authorization may include modules to break and mend the glass, and Patient may have a functionality to read patient medical history. These functionalities are defined as part of $\codebasel$ (Figure \ref{X}). Moreover, Figure \ref{X} describes the logging specification in $\callpols$ that is associated with the break-the-glass policy. In this clause, $t_0$ and $t_1$ are timestamps, and $t_1$ precedes $t_0$. $p$ refers to the patient identifier, and $u$ is the user identifier who breaks the glass and attempts to read the medical history of p later on. Additionally, logging is preconditioned on the fact that the glass is not mended in between the events of breaking the glass and gaining access to the patient medical data. 


\begin{figure}
\setlength{\fboxsep}{0pt}%%%%%%%%
\fbox{\tiny{
\begin{mathpar}
\forall t_0, \cdots, t_n, \mathit{xs_0}, \cdots, \mathit{xs_n} \, . \, 
\folp{Call}(t_0,A_0,B_0,\mathit{xs_0}) \bigwedge_{i=1}^{n} \big(\folp{Call}(t_i,A_i,B_i,\mathit{xs_i}) \wedge t_i < t_0 \big) \wedge \varphi(t_0, \cdots, t_n) \wedge \varphi'(\mathit{xs_0}, \cdots, \mathit{xs_n}) 
\bigwedge_{j=1}^{m} \big( \forall t'_j, \mathit{ys}_j \, . \, \psi_j(\mathit{xs}_0, \cdots, \mathit{xs}_n, \mathit{ys}_j) \wedge \psi'_j(t_0, \cdots, t_n, t'_j) \implies \neg \folp{Call}(t'_j, A'_j, B'_j, \mathit{ys}_j) \big) 
 \implies \folp{LoggedCall}(A_0,B_0,\mathit{xs_0})
\end{mathpar}
}}
\caption{$\callpols$ clause structure.} 
\label{fig:callpols}
\end{figure}


\begin{figure}
\setlength{\fboxsep}{0pt}%%%%%%%%
\fbox{\tiny{
\begin{mathpar}
\codebasel(\mathtt{Authorization})(\mathtt{breakTheGlass}) = [\mathtt{breakTheGlass}^{\mathtt{Authorization}} \triangleq P] \mbox{~for~some~} P 

\codebasel(\mathtt{Authorization})(\mathtt{mendTheGlass}) = [\mathtt{mendTheGlass}^{\mathtt{Authorization}} \triangleq P'] \mbox{~for~some~} P' 


\codebasel(\mathtt{Patient})(\mathtt{getMedicalHistory}) = [\mathtt{getMedicalHistory}^{\mathtt{Patient}} \triangleq Q] \mbox{~for~some~} Q  


\forall t_0, t_1, p, u \, . \, \folp{Call}(t_0,\mathtt{Patient}, \mathtt{getMedicalHistory},[p,u]) \wedge \folp{Call}(t_1,\mathtt{Authorization}, \mathtt{breakTheGlass},[u]) \wedge t_1 < t_0 \wedge (\forall t_2 \, . \, t_2 < t_0 \wedge t_1 < t_2 \implies \neg  \folp{Call}(t_1,\mathtt{Authorization}, \mathtt{mendTheGlass},[u])  ) \implies \folp{LoggedCall}(\mathtt{Patient}, \mathtt{getMedicalHistory},[p,u])

\end{mathpar}
}}
\caption{Example logging specification for an MRS.}
\label{fig:exm-logspec}
\end{figure}

\subsection{Target System Model} \label{sec:pi-log}
The instrumentation algorithm maps a $\srclang$ system to a target system, denoted by $\trglang$ system. Runtime environment of $\trglang$ includes a timing counter $t$, $\locprec$ that returns the set of logging preconditions transpired in a given agent, $\remprec$ that returns the set of logging preconditions that have transpired the triggers of a given agent, and $\logmap$ that stores the audit logs for a given agent. The preconditions in $\remprec$ must be communicated between a given agent and all triggers agents. Certain prefixes are added to $\trglang$ to facilitate storage and retrieval of information to these runtime components: 1) $\callev(A,B,\tilde{x})$ updates $\locprec(A)$ with predicate $\folp{Call}(t,A,B,\tilde{x})$, 2) $\addprecond(x,A)$ updates$\remprec(A)$ with precondition $x$, 3) $\sendprecond(x,A)$ converts $\locprec(A)$ to a transferable object and sends it though link $x$, and 4) $\emitev(A,B,\tilde{x})$ studies the derivability of $\folp{LoggedCall}(A,B,\tilde{x})$ and accordingly $\logmap(A)$ is updated with this predicate. 


\subsection{Instrumentation Algorithm} \label{sec:inst-alg}
Given a logging specification in $\ls \in \callpols$, algorithm $\rewrite$ translates a $\srclang$-system into a $\trglang$-system with concurrent audit logging capabilities to support $\ls$. In the following we briefly point out main aspects of how $\rewrite$ works. 

If $A_i$ is a logging event agent and $A_j$ is a positive or negative trigger agent, a fresh link $c_{ij}$ is added between them. This link is used to communicate locally transpired trigger events that are stored in trigger agents' $\locprec$ component. $\sendprecond$ and $\addprecond$ prefixes are used for this purpose.

$\rewrite$ inserts prefix $\callev$ before at the starting point of each negative/positive trigger sub-agent. This ensures storing trigger events in $\locprec$. 

$\callev$  is also inserted at the starting point of logging event sub-agents to capture logging events in $\locprec$ component of the corresponding agents. In addition, $\addprecond$ is inserted to notify all positive/negative trigger agents to send their trigger events through the established $c_{ij}$ link to the logging event tigger. The received trigger events are then stored in $\remprec$ component of the logging event agent. Finally, the logging event must check if the invocation of logging event sub-agent will be logged in $\logmap$. For this purpose, $\emitev$ is inserted.

Each positive/negative trigger $A_j$ must be able to respond to the requests by some logging event agent $A_i$ on the $c_{ij}$ link. To this end, $\rewrite$ introduces new sub-agent $D_{ij}$ in code base of $A_j$ that indefinitely listens on that link and returns the content of $\locprec$ on the same link. This is accomplished by $\sendprecond$ prefix.

As an example, having the definitions in Figure \ref{fig:exm-logspec} for an MRS, $\rewrite$ modifies the definition of sub-agents  $\mathtt{breakTheGlass}^{\mathtt{Authorization}}$, $\mathtt{mendTheGlass}^{\mathtt{Authorization}}$ and $\mathtt{getMedicalHistory}^{\mathtt{Patient}}$ by injecting proper prefixes at their starting points. In addition, a new link $c_{\mathtt{PA}}$ is introduced between $\mathtt{Authorization}$ and $\mathtt{Patient}$ agents, and a new sub-agent  $D_{\mathtt{PA}}^{\mathtt{Authorization}}$ is added that listens on $c_{\mathtt{PA}}$ indefinitely for requests from  $\mathtt{Patient}$ and responds with locally transpired trigger events, i.e., invocations to $\mathtt{breakTheGlass}$ (positive) and $\mathtt{mendTheGlass}$ (negative) in this example.


\begin{figure}
\setlength{\fboxsep}{0pt}%%%%%%%%
\fbox{\tiny{
\begin{mathpar}
\codebasel'(\mathtt{Authorization})(\mathtt{breakTheGlass}) = \big[\mathtt{breakTheGlass}^{\mathtt{Authorization}}(u) \triangleq \callev(\mathtt{Authorization},\mathtt{breakTheGlass},[u]).P \big]

\codebasel'(\mathtt{Authorization})(\mathtt{mendTheGlass}) = \big[\mathtt{mendTheGlass}^{\mathtt{Authorization}}(u) \triangleq \callev(\mathtt{Authorization},\mathtt{mendTheGlass},[u]).P' \big]

\codebasel'(\mathtt{Patient})(\mathtt{getMedicalHistory}) = \big[ \mathtt{getMedicalHistory}^{\mathtt{Patient}}(p,u) \triangleq \callev(\mathtt{Patient},\mathtt{getMedicalHistory},[p,u]).\outprefix{c_{\mathtt{PA}}}{}.c_{\mathtt{PA}}(f). \addprecond(f,\mathtt{Patient}). \newline \emitev(\mathtt{Patient},\mathtt{getMedicalHistory},[p,u]).Q \big]


\codebasel'(\mathtt{Authorization})(D_{\mathtt{PA}}) =  \big[ D{_{\mathtt{PA}}^{\mathtt{Authorization}}}(c_{\mathtt{PA}}) \triangleq 
c_{\mathtt{PA}}.\sendprecond(c_{\mathtt{PA}},\mathtt{Authorization}). D{_{\mathtt{PA}}^{\mathtt{Authorization}}}(c_{\mathtt{PA}}) \big]

\end{mathpar}
}}
\caption{Example Instrumentation of the MRS.}
\label{fig:exm-inst}
\end{figure}

