\section{Conclusion} \label{sec:conclusion}
In this paper, we have proposed an instrumentation tool for Java Spring based microservices to enforce audit requirements, whose logical specifications are not expressible by Horn clause logic. Our implementation tool is an extension of an earlier solution that supported specifications in Horn clauses. Our tool receives the specification in JSON format, along with the source code of microservices. It identifies the triggers and logging events in the application, and accordingly instruments them. As a case study, we discuss a microservices-based medical records system that lets users to deactivate controlling access to patient medical information in critical scenarios and activate it at a later stage. The correctness of the instrumentation tool relies on a formal model of the instrumentation that guarantees the generated logs to be necessary and sufficient.


%In this paper, we have proposed a tool, $\tool$, to instrument microservices-based applications that are deployed in Java Spring Framework for audit logging purposes. Our tool is based on an implementation model for concurrent systems that guarantees correctness of audit logging, using an information-algebraic semantic framework. $\tool$ receives the application source code, consisting of two or more microservices, along with a specification of audit logging requirements in JSON format. $\tool$ parses the JSON specification and extracts Horn clauses that are fed to a logic programming engine. $\tool$ instruments the microservices according to this specification. The instrumentation includes adding new repositories to the corresponding microservices, extending RESTful APIs on those microservices for logging-related communications, and weaving audit logging into the control flow of microservices using AspectJ. Our case study is a medical records system in which certain actions in authorization microservice may trigger logging events in access to patient medical data.

